\documentclass{beamer}
\usepackage{listings}
\lstset{
%language=C,
frame=single, 
breaklines=true,
columns=fullflexible
}
\usepackage{subcaption}
\usepackage{setspace}
\usepackage{url}
\usepackage{tikz}
\usepackage{tkz-euclide} % loads  TikZ and tkz-base
%\usetkzobj{all}
\usepackage[utf8]{inputenc}
\usepackage{longtable}
\usetikzlibrary{calc,math}
\usepackage{float}
\usetheme{Berlin}
\usepackage{graphicx}
\usecolortheme{beaver}


\newcommand\norm[1]{\left\lVert#1\right\rVert}
\renewcommand{\vec}[1]{\mathbf{#1}}
\usepackage[export]{adjustbox}
\usepackage[utf8]{inputenc}
\usepackage{amsmath}
%\usetheme{Boadilla}
\newcommand\mytextbullet{\leavevmode%
\usebeamertemplate{itemize item}\hspace{.5em}}

\bibliographystyle{IEEEtran}

\usepackage{color}

\title{Connecting Dory (Receiver) to Raspberry Pi 4}
\author{B603 Lab}
%{\and} \\
%\vspace{10pt}
%{Supervisor:} \\
%{Dr. GVV Sharma} }

\institute{Indian Institute of Technology, Hyderabad.}
\date{\today}

\begin{document}


\begin{frame}
\titlepage
\end{frame}

\section{Step 1}
\begin{frame}
\frametitle{Step 1}
\begin{columns}
\column{1\textwidth}
  \begin{enumerate}
  \item Install PuTTY: \\sudo apt-get install -y putty\\
  \vspace{10pt}
  \item Open Putty from the terminal using the command: putty
  
  \end{enumerate}
\end{columns}

\end{frame}

\section{Step 2}
\begin{frame}
\frametitle{Step 2}
\begin{columns}
\column{1\textwidth}
Open a terminal on RPi4 and type the command:
  \begin{itemize}
  \item  lsusb -t
  \vspace{10pt}
  \item dmesg $|$ grep tty  (From here check the Rx Dory is connected to which port of RPi 4)
  \vspace{10pt}
  \item lsusb /dev/tty* (Checking all the ports)
  
  \end{itemize}
\end{columns}
%\vspace{25px}



\end{frame}

\section{Step 3}
\begin{frame}
\frametitle{Step 3}
\begin{columns}
\column{1\textwidth}
In the PuTTy configuration,
  \begin{itemize}
  \item  Select the ‘Serial’ option.
  \vspace{10pt}
  \item Baud rate/speed = 115200.
  \vspace{10pt}
  \item Open.
  
  \end{itemize}
  
\end{columns}
%\vspace{25px}



\end{frame}

\section{Step 3}
\begin{frame}
\frametitle{Step 3}
\begin{columns}
\column{1\textwidth}

  \begin{figure}[h!]
  \centering
  \begin{subfigure}[b]{0.5\linewidth}
    \includegraphics[width=\linewidth]{./figs/5.png}
%    \caption{Coffee.}
  \end{subfigure}

  \caption{PUTTY Configuration}
%  \label{fig:axis}
\end{figure}
\end{columns}
%\vspace{25px}



\end{frame}

\section{Step 4}
\begin{frame}
\frametitle{Step 4}
\begin{columns}
\column{1\textwidth}

  \begin{itemize}
  \item  Make the transmitter Dory in ‘Fly’ mode.
  
  
  \end{itemize}
  \begin{figure}[h!]
  \centering
  \begin{subfigure}[b]{0.75\linewidth}
    \includegraphics[width=\linewidth]{./figs/2.png}
%    \caption{Coffee.}
  \end{subfigure}

  \caption{Serial Output of PUTTY for Rx. DORY}
%  \label{fig:axis}
\end{figure}
  
\end{columns}
%\vspace{25px}



\end{frame}
































\end{document}

